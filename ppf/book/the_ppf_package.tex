\chapter{The PPF package}\label{ch:the-ppf-package}

The source code accompanying this book implements a minimal library,
\verb|ppf|, for exploring financial modelling in Python. The sections
ahead outline the structure and ideas of the package.

The following is a first example of a financial program expressed in
Python -- the `Hello World' of Quantitative Analysis programs, that
is, the Black-Scholes formula for a European option on a single asset:

\begin{verbatim}
from math import log, sqrt, exp
from ppf.math import N

def black_scholes(S, K, r, sig, T, CP, *arguments, **keywords):
  """The classic Black and Scholes formula.

  >>> print black_scholes(S=42., K=40., r=0.1, sig= 0.2, T=0.5, CP=CALL)
  4.75942193531

  >>> print black_scholes(S=42., K=40., r=0.1, sig= 0.2, T=0.5, CP=PUT)
  0.808598915338

  """
  d1 = (log(S/K) + (r + 0.5*(sig*sig))*T)/(sig*sqrt(T))
  d2 = d1 - sig*sqrt(T)

  return CP*S*N(CP*d1) - CP*K*exp(-r*T)*N(CP*d2)

CALL, PUT = (1, -1)

def _test():
  import doctest
  doctest.testmod()

if __name__ == '__main__': _test()

\end{verbatim}
%% The above code will print
%%4.75942
%%0.808599

\section{PPF Topology}

The \verb|ppf| library is a Python package containing a family of
sub-packages. The \verb|black_scholes| function listed above is housed
in the \verb|ppf.core| subpackage. The topology of \verb|ppf| is as
follows:
\begin{verbatim}
  ppf/
    com/
    core/
    date_time/
    market/
    math/
    model/
      hull_white/
        lattice/
        monte_carlo/
    pricer/
      payoffs/
    test/
    utility/
\end{verbatim}
Here is a brief summary of the natures and main roles of each the
\verb|ppf| sub-packages:
\begin{description}
\item[com] COM servers wrapping \verb|ppf| market, trade and pricing
functionality (see chapter \ref{ch:python-excel-integration}).
\item[core] Types and functions relating to the representation of
financial quantities such as flows and LIBOR rates.
\item[date\_time] Date and time manipulation and computations.
\item[market] Types and functions for the representation of common
curves and surfaces that arise in financial programming such as
discount factor curves and volatility surfaces.
\item[math] General mathematical algorithms.
\item[model] Code specific to implementing numerical pricing models.
\item[pricer] Types and functions for the purpose of valuing financial
structures.
\item[test] The \verb|ppf| unit-test suite.
\item[utility] Utilities of a less numerical, general nature such as
algorithms for searching and sorting.
\end{description}

\section{Unit-testing}

Code in the \verb|ppf| library employs two approaches to testing:
interactive Python session testing using the \verb|doctest| module and
formalized unit-testing using the \verb|PyUnit| module. Both of these
testing frameworks are part of the Python standard libraries.

\subsection{doctest}

The way that the \verb|doctest| module works is to search a module for
pieces of text that look like interactive Python sessions, and then
to execute those sessions to verify that they work as expected. In
this way then, \verb|ppf| modules come with a form of tutorial-like
executable documentation:
\begin{verbatim}
C:\Python25\lib\site-packages\ppf\core>python black_scholes.py -v
python black_scholes.py -v
Trying:
    print black_scholes(S=42., K=40., r=0.1, sig= 0.2, T=0.5, CP=CALL)
Expecting:
    4.75942193531
ok
Trying:
    print black_scholes(S=42., K=40., r=0.1, sig= 0.2, T=0.5, CP=PUT)
Expecting:
    0.808598915338
ok
2 items had no tests:
    __main__
    __main__._test
1 items passed all tests:
   2 tests in __main__.black_scholes
2 tests in 3 items.
2 passed and 0 failed.
Test passed.
\end{verbatim}

\subsection{PyUnit}

A full suite of unit-tests for all modules in the \verb|ppf| package
is provided in the \verb|ppf.test| sub-package. The tests can be run
module-by-module or, to execute all tests in one go, a driver
`test\_all.py' is provided:
\begin{verbatim}
C:\Python25\Lib\site-packages\ppf\test>python test_all.py --verbose
python test_all.py --verbose
test_call (test_core.black_scholes_tests) ... ok
test_put (test_core.black_scholes_tests) ... ok
test (test_core.libor_rate_tests) ... ok
    .
    .
    .
test_upper_bound (test_utility.bound_tests) ... ok
test_equal_range (test_utility.bound_tests) ... ok
test_bound (test_utility.bound_tests) ... ok
test_bound_ci (test_utility.bound_tests) ... ok

----------------------------------------------------------------------
Ran 51 tests in 25.375s

OK
\end{verbatim}

\section{Building and Installing PPF}

In this section we look at what it takes to build and install the
\verb|ppf| package.

\subsection{Prerequisites and dependencies}

\verb|ppf| is composed of a mixture of pure Python modules underlied
by some supporting extension modules implemented in standard
C++. Accordingly, to build and install \verb|ppf| requires a modern C++
compiler. The C++ extension modules have some library dependencies of
their own, notably the Boost C++ libraries and the Blitz++ C++
library. Instructions for downloading and installing the Boost C++
libraries can be found at \verb|http://www.boost.org| and instructions
for Blitz++ can be found at
\verb|http://www.oonumerics.org|. Naturally, an installation of Python
is also required. On Windows, the authors favour the freely available
ActiveState Python distribution, see \verb|http://www.activestate.com|
for download and installation details. Also required on the Python
side for \verb|ppf| is an installation of the NumPy package, see
\verb|http://www.scipy.org| for download and installation details.

\subsection{Building the C++ extension modules}

The \verb|ppf| C++ extension modules are most conveniently built using
the Boost.Build system\footnote{See
http://www.boost.org/doc/tools/build/index.html} a copy of which is
included with the \verb|ppf| sources. Also provided with the
\verb|ppf| sources for the convenience of Windows users is a pre-built
executable `bjam.exe'. Although these notes will become a little
Windows-centric at this point, the basic principles will hold for *NIX
users also. On Windows, the \verb|ppf| package has been successfully
built and tested with the Microsoft Visual Studio C++ compiler
versions 7.1, 8.0 (express edition), 9.0 (express edition),
mingw/gcc-3.4.5\footnote{Minimalist GNU for Windows -- see
http://www.mingw.org.}, mingw/gcc-4.3.0 with Python versions 2.4 and
2.5, Boost versions 1.33.1, 1.34.0, 1.35, 1.36, 1.37 and Blitz++
version 0.9.  The \verb|ppf| package has also been built and tested on
the popular Linux-based operating system, Ubuntu-8.04.1 with Boost
version 1.36.0, Blitz++ version 0.9 and gcc-4.2.3.

In the remainder of this section without loss of generality, we will
assume a Windows operating system, Blitz++ version 0.9, the
ActiveState distribution of Python version 2.5 and Boost version 1.36.

\subsubsection{Build instructions}
\begin{itemize}
  \item{Prerequisites}
  \begin{itemize}
    \item{Copy \verb|c:/path/to/ppf/ext/bjam.exe| to somewhere in your \verb|%PATH%|}
    \item{Install}
    \begin{itemize}
      \item{Blitz++-0.9}
      \item{Boost-1.36}
      \item{ActiveState Python 2.5}
      \item{NumPy for Python 2.5 (version 1.0.4 or 1.1.0)}
    \end{itemize}
    \item{Edit as appropriate for your site}
    \begin{itemize}
      \item{\verb|c:/path/to/ppf/ext/build/user-config.jam|}
      \item{\verb|c:/path/to/ppf/ext/build/site-config.jam|}
    \end{itemize}
  \end{itemize}
  \item{Build}
  \begin{itemize}
    \item{\verb@c:/path/to/ppf>cd ext&&bjam [debug|release]@\\This will create:}
    \begin{itemize}
      \item{\verb|c:/path/to/ppf/ppf/math/ppf_math.pyd| and}
      \item{\verb|c:/path/to/ppf/ppf/date_time/ppf_date_time.pyd|}
    \end{itemize}
  \end{itemize}
\end{itemize}

\subsection{Installing the PPF package}

Assuming the steps of the previous section have been performed,
installation of the \verb|ppf| package which relies on the standard Python
Distutils package is very simple.
\begin{itemize}
  \item{Install}
  \begin{itemize}
    \item{\verb|c:/path/to/ppf>python setup.py install|}
  \end{itemize}
\end{itemize}
which will copy the \verb|ppf| package to the standard Python
installation location (\verb|c:/python25/lib/site-packages/ppf|).

\subsection{Testing a PPF installation}

The easiest way to verify a \verb|ppf| installation is to run the
\verb|ppf| unit-test suite.

\begin{itemize}
  \item{Test}
  \begin{itemize}
  \item{\verb|c:/python25/lib/site-packages/ppf/test>python test_all.py --verbose|}
  \end{itemize}
\end{itemize}
